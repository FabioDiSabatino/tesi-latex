\chapter*{Sommario}
\markboth{Sommario}{Sommario}
\addcontentsline{toc}{chapter}{Sommario}
\thispagestyle{empty}

Lo scopo di questa tesi è lo sviluppo di un'applicazione mobile cross-platform geolocalizzata, per contesti di disaster management.\\
 Gli utenti possono segnalare la posizione delle emergerze e il proprio stato fisico; queste informazioni vengono inviate ad un main-server, il quale sarà utilizzato dai soccorritori per gestire il disastro e coordinare al meglio le operazioni di salvataggio. Inoltre attraverso l'applicazione è possibile visualizzare lo stato dei propri famigliari, dei luoghi d'interesse e quello generale dell'area colpita.\\
 Per realizzare l'applicazione si sono utilizzati i seguenti framework: Phonegap per renderla cross-platform, Ratchet per realizzare l'interfaccia e la libreria Javascript Leaflet per la mappa interattiva. Le posizioni degli utenti e degli eventi sono mappate in una griglia trasparente all'utente, utilizzando un algoritmo appositamente ideato.
\\
\\

\noindent
\begin{Huge}
\textbf{Abstract}
\end{Huge}
\\
\\
\noindent
The purpose of this thesis is the development of a cross-platform geolocalized mobile application for disaster management.\\
 Users can signal the position of emergencies and send their physical status; these information are sent to a main-server that will be used by rescuers to manager and optimize the rescue operations. Through this application users can also check the status of their family and other point of interest, moreover they can visualize the general state of the area affected by disaster.\\
 To develop this application, we have used the following framework: Phonegap for the cross-platform requirement, Ratchet to implement the interface and Leaflet library for the interactive map. User's and event's position are match within a trasparent grid, using an algorithm ideate.

