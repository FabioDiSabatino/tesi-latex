\appendix
\chapter{Configurazioni generali scenari}
\label{app:a}


\section{1}
\label{app:conf_lxc_nodo1}

\lstset{language=BASH}
\begin{lstlisting}
lxc.utsname=node-1
lxc.network.type=veth
lxc.network.name=eth1
lxc.network.flags=up
lxc.network.link=emanenode0
lxc.network.hwaddr=02:00:01:01:00:01
lxc.network.ipv4=10.99.0.1/24
lxc.network.veth.pair=veth1.1

lxc.network.type=veth
lxc.network.name=eth2
lxc.network.hwaddr=02:00:01:02:00:01
lxc.network.veth.pair=veth1.2

lxc.network.type = empty
lxc.network.flags=up

lxc.console = none
lxc.tty = 1
lxc.pts = 128
lxc.cgroup.devices.allow = a
lxc.mount.entry = /home/emanuele/Tesi/freespace/var/lib/emanenode none defaults,bind 0 0

lxc.aa_profile = unconfined
\end{lstlisting}

\section{2}
\label{app:conf_nem}

\lstset{language=XML}
\begin{lstlisting}
<?xml version="1.0" encoding="UTF-8"?>
<!DOCTYPE nem SYSTEM "file:///usr/share/emane/dtd/nem.dtd">
<nem name="NEM" type="structured">
  <transport definition="transvirtual.xml"/>
  <mac definition="ieee80211abgmac.xml"/>
  <phy definition="phy.xml"/>
  <shim definition="commeffectshim.xml"/>
</nem>
\end{lstlisting}

\section{3}
\label{app:virtual_transport}

\lstset{language=XML}
\begin{lstlisting}
<?xml version="1.0" encoding="UTF-8"?>
<!DOCTYPE transport SYSTEM "file:///usr/share/emane/dtd/transport.dtd">
<transport name="Tap transport" library="transvirtual">
	<param name="flowcontrolenable" value="off"/>
	<param name="devicepath" value="/dev/net/tun"/>
</transport>
\end{lstlisting}

\section{Configurazione XML IEEE 802.11abg MAC Layer}
\label{app:80211}

\lstset{language=XML}
\begin{lstlisting}
<?xml version="1.0" encoding="UTF-8"?>
<!DOCTYPE mac SYSTEM "file:///usr/share/emane/dtd/mac.dtd">
<mac name="IEEE 802.11 MAC" library="ieee80211abgmaclayer">
  <param name="mode"                  value="3"/>
  <param name="enablepromiscuousmode" value="off"/>
  <param name="unicastrate"           value="12"/>
  <param name="multicastrate"         value="12"/>
  <param name="rtsthreshold"          value="0"/>
  <param name="pcrcurveuri"      value="ieee80211pcr.xml"/>
  <param name="flowcontrolenable"     value="off"/>
  <param name="flowcontroltokens"     value="10"/>
  <param name="radiometricenable"     value="on"/>
</mac>
\end{lstlisting}

\section{Configurazione XML PHY Layer (freespace)}
\label{app:freespace}

\lstset{language=XML}
\begin{lstlisting}
<?xml version="1.0" encoding="UTF-8"?>
<!DOCTYPE phy SYSTEM "file://usr/share/emane/dtd/phy.dtd">
<phy name="Phy">
    <param name="bandwidth"             value="20M"/>
    <param name="propagationmodel"      value="freespace"/>
    <param name="noisemode"             value="outofband"/>
    <param name="systemnoisefigure"     value="4.0"/>
    <param name="txpower"               value="0.0"/>
    <param name="frequency"             value="2.347G"/>
    <param name="subid"                 value="1"/>
</phy>
\end{lstlisting}

\section{Configurazione XML PHY Layer (2-ray)}
\label{app:2-ray}

\lstset{language=XML}
\begin{lstlisting}
<?xml version="1.0" encoding="UTF-8"?>
<!DOCTYPE phy SYSTEM "file://usr/share/emane/dtd/phy.dtd">
<phy name="Phy">
    <param name="bandwidth"             value="20M"/>
    <param name="propagationmodel"      value="2ray"/>
    <param name="noisemode"             value="outofband"/>
    <param name="systemnoisefigure"     value="4.0"/>
    <param name="txpower"               value="0.0"/>
    <param name="frequency"             value="2.347G"/>
    <param name="subid"                 value="1"/>
</phy>
\end{lstlisting}

\section{Comm Effect Shim Layer}
\label{app:shim_layer}

\lstset{language=XML}
\begin{lstlisting}
<?xml version="1.0" encoding="UTF-8"?>
<!DOCTYPE shim SYSTEM "file:///usr/share/emane/dtd/shim.dtd">
<shim library="commeffectshim">
  <param name="defaultconnectivitymode" value="off" />
  <param name="enablepromiscuousmode" value="off" />
</shim>
\end{lstlisting}

\section{NEM Platform Server nodo-1}
\label{app:platform_server}

\lstset{language=XML}
\begin{lstlisting}
<?xml version="1.0" encoding="UTF-8"?>
<!DOCTYPE platform SYSTEM "file:///usr/share/emane/dtd/platform.dtd">
<platform>
  <param name="otamanagerchannelenable" value="on"/>
  <param name="otamanagerdevice" value="eth1"/>
  <param name="otamanagergroup" value="224.1.2.8:45702"/>
  <param name="eventservicegroup" value="224.1.2.8:45703"/>
  <param name="eventservicedevice" value="eth1"/>
  <param name="controlportendpoint" value="0.0.0.0:47000"/>

  <nem id="1" definition="nem.xml">
    <transport definition="transvirtual.xml">
      <param name="address" value="10.100.0.1"/>
      <param name="mask" value="255.255.255.0"/>
    </transport>
    </nem>
</platform>
\end{lstlisting}

\section{Configurazione protocollo di routing OLSR}
\label{app:routing}

\lstset{language=BASH}
\begin{lstlisting}
DebugLevel	0

# IP version to use (4 or 6)

IpVersion	4

LockFile "/var/lib/emanenode/persist/1/var/run/olsrd.lock"

# Clear the screen each time the internal state changes

ClearScreen     yes

LinkQualityAging 0.2

Interface "emane0" "emane1"
{
 HelloInterval 1.0
 HelloValidityTime 5.0
}

LoadPlugin "olsrd_txtinfo.so.0.1"
{
PlParam "accept" "0.0.0.0"
}
\end{lstlisting}

\section{Configurazione XML di emaneeventservice}
\label{app:emaneeventservice}

\lstset{language=XML}
\begin{lstlisting}
<?xml version="1.0" encoding="UTF-8"?>
<!DOCTYPE eventservice SYSTEM "file:///usr/share/emane/dtd/eventservice.dtd">
<eventservice>
  <param name="eventservicegroup" value="224.1.2.8:45703"/>
  <param name="eventservicedevice" value="emanenode0"/>
  <generator definition="eelgenerator.xml"/>
</eventservice>
\end{lstlisting}

\section{Configurazione XML EEL Generator}
\label{app:eel_generator}

\lstset{language=XML}
\begin{lstlisting}
<?xml version="1.0" encoding="UTF-8"?>
<!DOCTYPE eventgenerator SYSTEM "file:///usr/share/emane/dtd/eventgenerator.dtd">
<eventgenerator library="eelgenerator">
  <param name="inputfile" value="../scenario/scenario.eel" />
  <paramlist name="loader">
    <item value="commeffect:eelloadercommeffect:delta"/>
    <item value="location,velocity,orientation:eelloaderlocation:delta"/>
    <item value="pathloss:eelloaderpathloss:delta"/>
    <item value="antennaprofile:eelloaderantennaprofile:delta"/>
  </paramlist>
</eventgenerator>
\end{lstlisting}

\section{Configurazione XML Event Daemon}
\label{app:event_daemon}

\lstset{language=XML}
\begin{lstlisting}
<?xml version="1.0" encoding="UTF-8"?>
<!DOCTYPE eventdaemon SYSTEM "file:///usr/share/emane/dtd/eventdaemon.dtd">
<eventdaemon nemid="1">
  <param name="eventservicegroup" value="224.1.2.8:45703"/>
  <param name="eventservicedevice" value="eth1"/>
  <agent definition="gpsdlocationagent1.xml"/>
</eventdaemon>
\end{lstlisting}

\section{File EEL}
\label{app:eel_file}

\lstset{language=BASH}
\begin{lstlisting}
0.0  nem:1 commeffect nem:2,0,0,0,0,0,0
0.0  nem:2 commeffect nem:1,0,0,0,0,0,0
0.0  nem:2 commeffect nem:3,0,0,0,0,0,0
0.0  nem:3 commeffect nem:2,0,0,0,0,0,0
0.0  nem:3 commeffect nem:4,0,0,0,0,0,0
0.0  nem:4 commeffect nem:3,0,0,0,0,0,0
0.0  nem:4 commeffect nem:5,0,0,0,0,0,0
0.0  nem:5 commeffect nem:4,0,0,0,0,0,0
0.0  nem:5 commeffect nem:6,0,0,0,0,0,0
0.0  nem:6 commeffect nem:5,0,0,0,0,0,0
0.0  nem:6 commeffect nem:7,0,0,0,0,0,0
0.0  nem:7 commeffect nem:6,0,0,0,0,0,0
0.0  nem:8 commeffect nem:1,0,0,0,0,0,0
0.0  nem:1 commeffect nem:8,0,0,0,0,0,0
0.0  nem:8 commeffect nem:2,0,0,0,0,0,0
0.0  nem:2 commeffect nem:8,0,0,0,0,0,0
0.0  nem:8 commeffect nem:3,0,0,0,0,0,0
0.0  nem:3 commeffect nem:8,0,0,0,0,0,0
0.0  nem:8 commeffect nem:4,0,0,0,0,0,0
0.0  nem:4 commeffect nem:8,0,0,0,0,0,0
0.0  nem:8 commeffect nem:5,0,0,0,0,0,0
0.0  nem:5 commeffect nem:8,0,0,0,0,0,0
0.0  nem:8 commeffect nem:6,0,0,0,0,0,0
0.0  nem:6 commeffect nem:8,0,0,0,0,0,0
0.0  nem:8 commeffect nem:7,0,0,0,0,0,0
0.0  nem:7 commeffect nem:8,0,0,0,0,0,0
0.0  nem:1 location gps 40.000000,-74.480000,1.000000
0.0  nem:2 location gps 40.000000,-74.480700,1.000000
0.0  nem:3 location gps 40.000000,-74.481400,1.000000
0.0  nem:4 location gps 40.000000,-74.482100,1.000000
0.0  nem:5 location gps 40.000000,-74.482800,1.000000
0.0  nem:6 location gps 40.000000,-74.483500,1.000000
0.0  nem:7 location gps 40.000000,-74.484200,1.000000
0.0  nem:8 location gps 40.000400,-74.478600,1.000000
1.0  nem:8 location gps 40.000400,-74.478614,1.000000
2.0  nem:8 location gps 40.000400,-74.478628,1.000000
3.0  nem:8 location gps 40.000400,-74.478642,1.000000
\end{lstlisting}

\section{Configurazione XML GPS}
\label{app:gps}

\lstset{language=XML}
\begin{lstlisting}
<?xml version="1.0" encoding="UTF-8"?>
<!DOCTYPE eventagent SYSTEM "file:///usr/share/emane/dtd/
		eventagent.dtd">
<eventagent library="gpsdlocationagent">
  <param name="gpsdconnectionenabled" value="no"/>
  <param name="pseudoterminalfile"
         value="persist/1/var/run/gps.pty"/>
</eventagent>
\end{lstlisting}

\chapter{Script}
\label{app:script}

Qui vengono presentati gli script da me realizzati per raggiungere scopi diversi per una buona riuscita delle emulazioni e delle misure effettuate.

\section{Script generatore EEL}
\label{app:script_eel_generator}

\lstset{language=PYTHON}
\begin{lstlisting}
handle = open("./scenario.eel","w");

nem = 1;
while (nem < 7):
	nemsucc=nem+1;
	commeffect= "0.0  nem:"+str(nem)+" commeffect nem:"+str(nemsucc)+",0,0,0,0,0,0\n";
	commeffect1= "0.0  nem:"+str(nemsucc)+" commeffect nem:"+str(nem)+",0,0,0,0,0,0\n";
	handle.write(commeffect);
	handle.write(commeffect1);
	nem=nem+1;

nem=1;
while (nem<=7) :
	commeffect = "0.0  nem:8 commeffect nem:"+str(nem)+",0,0,0,0,0,0\n0.0  nem:"+str(nem)+" commeffect nem:8,0,0,0,0,0,0\n";
	handle.write(commeffect);
	nem=nem+1;

i=-74.48;

nem = 1;
while (nem <= 7) :
	location = "0.0  nem:"+str(nem)+" location gps 40.000000,"+str("%.6f"% i)+",1.000000\n"
	handle.write(location);
	i = i - 0.0007;
	nem=nem+1;

i=-74.478586;
massimo=-74.4856;
sec=0;

while (i>massimo) :
	i=i-0.000014;
	handle.write(str(sec)+".0  nem:8 location gps 40.000400,"+str("%.6f"% i)+",1.000000\n");
	sec=sec+1;

handle.close();
\end{lstlisting}

\section{Script avvio emulazione}
\label{app:script_avvio}

Script da lanciare:

\lstset{language=BASH}
\begin{lstlisting}
#!/bin/bash -

prefix=node-
bridge=emanenode0
nodevolume=/var/lib/emanenode
nodecount=8

mkdir -p persist/host/var/log
mkdir -p persist/host/var/run
rm -f persist/host/var/log/*

echo "Creating bridge: $bridge"
                
brctl addbr $bridge
                
ip addr add 10.99.0.100/24 dev $bridge
                
ip link set $bridge up
                
iptables -I INPUT -i $bridge -j ACCEPT
                
iptables -I FORWARD -i $bridge -j ACCEPT

# disable realtime scheduling contraints
sysctl kernel.sched_rt_runtime_us=-1


for nodeid in $(seq 1 $nodecount)
        do
            mkdir -p persist/$nodeid/var/log
            mkdir -p persist/$nodeid/var/run
            rm -f persist/$nodeid/var/log/*
            rm -f persist/$nodeid/var/run/*

            name=$prefix$nodeid
            hex=$(printf "%02x" $nodeid)

            cat <<EOF > persist/$nodeid/var/run/lxc.conf.$nodeid
lxc.utsname=$name
lxc.network.type=veth
lxc.network.name=eth1
lxc.network.flags=up
lxc.network.link=$bridge
lxc.network.hwaddr=02:00:$hex:01:00:01
lxc.network.ipv4=10.99.0.$nodeid/24
lxc.network.veth.pair=veth$nodeid.1

lxc.network.type=veth
lxc.network.name=eth2
lxc.network.hwaddr=02:00:$hex:02:00:01
lxc.network.veth.pair=veth$nodeid.2

lxc.network.type = empty
lxc.network.flags=up

lxc.console = none
lxc.tty = 1
lxc.pts = 128
lxc.cgroup.devices.allow = a
lxc.mount.entry = $(pwd) $nodevolume none defaults,bind 0 0

lxc.aa_profile = unconfined
EOF

lxc-execute -f persist/$nodeid/var/run/lxc.conf.$nodeid \
                -n $name \
                -o persist/$nodeid/var/log/
                   lxc-execute.log.$nodeid \
                -l DEBUG \
                -- \
                ./carica_servizi $nodeid &

            while (! test -f persist/$nodeid/var/log/lxc-execute.log.$nodeid)
            do
                sleep 1
            done

        done

emaneeventservice eventservice.xml -l 3 > persist/host/var/log/emaneeventservice.log
\end{lstlisting}

\section{Script caricatore di servizi}
\label{app:script_richiamato}

Questo script viene richiamato all'interno del precedente (Appendice \ref{app:script_avvio}) al fine di caricare tutti i servizi sui vari nodi dell'emulazione. \\

\lstset{language=BASH}
\begin{lstlisting}
#!/bin/bash -

export PATH=$PATH:/sbin:/usr/sbin

nodeId=$1

# fix devpts
mount /dev/pts -o remount,rw,mode=620,ptmxmode=666,gid=5

# Si avvia sshd sul nodo in modo da poter entrare tramite ssh
/usr/sbin/sshd -o "PidFile=persist/$nodeId/var/run/ssh.pid"

# Si avvia emane sul nodo specificando il file di configurazione, il file di log, il file del pid e il file dell'uuid
emane platform$nodeId.xml -r -d -l 4 -f persist/$nodeId/var/log/emane.log \
--pidfile persist/$nodeId/var/run/emane.pid --uuidfile persist/$nodeId/var/run/emane.uuid      

# Avviare gli eventi sui nodi
rm -f persist/$nodeId/var/run/gps.pty

echo "Starting emaneeventd: eventdaemon$nodeId.xml"

emaneeventd -r -d -l 3 eventdaemon$nodeId.xml -f persist/$nodeId/var/log/emaneeventd.log \
--pidfile persist/$nodeId/var/run/emaneeventd.pid --uuidfile persist/$nodeId/var/run/emaneeventd.uuid

echo "Waiting for GPS Location PTY: persist/$nodeId/var/run/gps.pty"

while (! test -f persist/$nodeId/var/run/gps.pty)
do
	sleep 1
done

#Imposta il livello di log al massimo (DEBUG)
#emanesh node-$nodeId loglevel 4

# Avvio del gps
echo "Starting gpsd: persist/$nodeId/var/run/gps.pty"

gpsd -P persist/$nodeId/var/run/gpsd.pid -G -n -b `cat persist/$nodeId/var/run/gps.pty`

# Avvia olsrd
olsrd -f "routing$nodeId.conf" -d 3 > persist/$nodeId/var/log/routing$nodeId.log
\end{lstlisting}

\section{Script arresto emulazione}
\label{app:script_arresto}

\lstset{language=BASH}
\begin{lstlisting}
#!/bin/bash -

prefix=node-
bridge=emanenode0
nodevolume=/var/lib/emanenode
bridgecontrol=1
nodecount=8

check_bridge()
{
    return $(brctl show | awk "/^$1[\t\s]+/{exit 1}") 
}

for nodeid in $(seq 1 $nodecount)
        do
            echo "Stopping lxc instance: $prefix$nodeid"

            lxc-stop -n $prefix$nodeid

        done
        
        if [ -f persist/host/var/run/emaneeventservice.pid ]
        then
            kill -QUIT $(cat persist/host/var/run/emaneeventservice.pid)
            rm -f persist/host/var/run/emaneeventservice.pid
        fi
                
        if [ -f persist/host/var/run/emane.pid ]
        then
            kill -QUIT $(cat persist/host/var/run/emane.pid)
            rm -f persist/host/var/run/emane.pid
        fi

        if [ -f persist/host/var/run/transportdaemon.pid ]
        then
            kill -QUIT $(cat persist/host/var/run/transportdaemon.pid)
            rm -f persist/host/var/run/transportdaemon.pid
        fi
        
        if [[ $bridgecontrol -ne 0 ]]
        then
            if (! check_bridge $bridge)
            then
                echo "Removing bridge: $bridge"
                
                ip link set $bridge down
                
                brctl delbr $bridge
                
                iptables -D INPUT -i $bridge -j ACCEPT
                
                iptables -D FORWARD -i $bridge -j ACCEPT
            fi
        fi

        sleep 2

        for vif in $(ip link show | awk -F : '/veth[0-9]+\.[0-9]/{print $2}')
        do
            echo "Performing extra cleanup of vif $vif"
            ip link del dev $vif 2>&1 > /dev/null
        done

        # paranoia - make sure everything is down
        for i in $(ps ax | awk '/emaneeventservic[e] /{print $1}')
        do
            echo "Performing extra cleanup of emaneeventservice [$i]"
            kill -9 $i;
        done

        for i in $(ps ax | awk '/emanetransport[d] /{print $1}')
        do
            echo "Performing extra cleanup of emanetransportd [$i]"
            kill -9 $i;
        done    

        for i in $(ps ax | awk '/eman[e] /{print $1}')
        do
            echo "Performing extra cleanup of emane [$i]"
            kill -9 $i;
        done  
\end{lstlisting}

\section{Script raccolta dati}
\label{app:script_raccolta}

Questo script php è stato realizzato per raccogliere (tramite l'utilizzo di Espressioni Regolari) i dati relativi alla potenza di ricezione del nodo 8 dal file di log di EMANE e le misurazioni fatte dal protocollo di routing relative all'ETX del nodo 8 rispetto agli altri nodi. \\

\lstset{language=PHP}
\begin{lstlisting}
<?php
$handle = fopen("./persist/8/var/log/emane.log","r");
$riga_emane = fgets($handle);
$primo = true;
while (!feof($handle)) {
	if (preg_match("/IEEEMACLayer::processUpstreamPacket: src [[:digit:]]{1}/", $riga_emane)) {
		$riga_emane1 = fgets($handle);
		if (preg_match("/IEEEMACLayer::processUpstreamPacket .+power: -[[:digit:]]{2}\.[[:digit:]]{2}/", $riga_emane1)) {
			$ora = substr($riga_emane, 0,2);
			$minuti = substr($riga_emane, 3,2);
			$secondi = substr($riga_emane, 6,2);
			$totale_secondi = $secondi + $minuti*60 + $ora*3600;
			if ($primo) {
				$inizio_emulazione = $totale_secondi;
				$primo = false;
			}
			$tempo = $totale_secondi - $inizio_emulazione;
			$src = substr($riga_emane, 72,1);
			$db = substr($riga_emane1, 161,7);
			$db = round($db,2);
			$db = substr($db,0,3).",".substr($db, 4);
			$power[$tempo][$src] = $db;
		} else {
			$riga_emane = $riga_emane1;
		}
	} else {
		$riga_emane = fgets($handle);
	}
}
fclose($handle);
ksort($power);
foreach ($power as $key => $value) {
	ksort($power[$key]);
}
$dati_estratti = fopen("./grafici/dati_estratti_power.csv", "w+");
foreach ($power as $key => $value) {
	$da_scrivere = $key;
	for ($i=1;$i<=7;$i++) {
		if (isset($value[$i])) {
			$da_scrivere = $da_scrivere.";".$value[$i];
		} else {
			$da_scrivere = $da_scrivere.";";
		}
	}
	$da_scrivere = $da_scrivere."\n";
	fwrite($dati_estratti,$da_scrivere);
}
fclose($dati_estratti);

$routing = fopen("./persist/8/var/log/routing8.log","r");
$riga_routing = fgets($routing);
while (!feof($routing)) {
	if (preg_match("/--- .+LINKS/", $riga_routing)) {
		$ora = substr($riga_routing, 4,2);
		$minuti = substr($riga_routing, 7,2);
		$secondi = substr($riga_routing, 10,2);
		$totale_secondi = $secondi + $minuti*60 + $ora*3600;
		$tempo = $totale_secondi - $inizio_emulazione;
		$riga_routing1 = fgets($routing);
		$riga_routing1 = fgets($routing);
		$riga_routing1 = fgets($routing);
		while (preg_match("/10\.100\.0\.[[:digit:]]/", $riga_routing1)) {
			$src = substr($riga_routing1, 9,1);
			$val = substr($riga_routing1, 39);
			if (trim($val) == "INFINITE") {
				$val = 0;
			} else {
				$val = explode(".", $val);
				$val = trim($val[0]).",".trim($val[1]);
			}
			if ($val != 0) {
				$etx[$tempo][$src] = $val;
			}
			$riga_routing1 = fgets($routing);
		}
		$riga_routing = $riga_routing1;
	} else {
		$riga_routing = fgets($routing);
	}
}
fclose($routing);
ksort($etx);
$time = array();
foreach ($etx as $key => $value) {
	ksort($etx[$key]);
	for ($i=1;$i<=7;$i++) {
		if (isset($value[$i])) {
			$time[$i] = $key;
		}
	}
}
foreach ($time as $key => $value) {
	$etx[$value+1][$key] = 25;
}
$dati_estratti = fopen("./grafici/dati_estratti_etx.csv", "w+");
foreach ($etx as $key => $value) {
	$da_scrivere = $key;
	for ($i=1;$i<=7;$i++) {
		if (isset($value[$i])) {
			$da_scrivere = $da_scrivere.";".$value[$i];
		} else {
			$da_scrivere = $da_scrivere.";";
		}
	}
	$da_scrivere = $da_scrivere."\n";
	fwrite($dati_estratti,$da_scrivere);
}
fclose($dati_estratti);
?>
\end{lstlisting}