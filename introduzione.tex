\chapter*{Introduzione}
\markboth{Introduzione}{Introduzione}
\addcontentsline{toc}{chapter}{Introduzione}
\thispagestyle{empty}

Durante una lezione del corso di progettazione di sistemi interattivi, tenuto dalla prof.ssa Laura Tarantino, fu chiesto a tutti gli studenti cosa ritenessero più tecnologico tra un libro e il proprio laptop. Una domanda apparentemente banale alla quale tutti gli studenti, me compreso, hanno risposto: laptop.\\
Ci venne spiegato che incosciamente siamo portati a ritenere un oggetto tecnologico se questo è stato inventanto dopo la nostra nascita o comunque in epoca recente.\\
Il fatto di non essere riuscito a rispondere correttamente a questa domanda mi ha fatto riflettere, come può un ingegnere non sapere per certo cosa sia tecnologico? Quando guardando la pubblicità dell'ultimo \textit{smartwatch} la risposta mi è parsa ovvia: negli ultimi decenni, il concetto di tecnologia è stato manipolato, trasformato e usato per fini puramente commerciali, inculcando nelle nostre menti un significato sbagliato di questo nobile concetto. 
\textbf{La tecnologia è progresso, è l'insieme di tutti gli oggetti e studi che in qualche modo contribuiscono a migliorare la nostra vita}.\\
Non ho nulla contro chi da la possibilità di parlare con una persona usando un orologio e condivido la necessità di realizzare profitti; piuttosto contesto l'allocazione totale della forza lavoro e delle risorse per la produzione di \textbf{tecnologia superflua}.\\
Come ingegnere (aspirante), credo di avere l'obbligo morale di contribuire al progresso tecnologico (o almeno provarci), nel rispetto di chi in passato ha dedicato la sua vita affinché oggi potessimo vivere meglio. \\
Concludo questa prefazione, prima di tediare il lettore, dicendo che non ho accettato questa tesi al fine unico di conseguire il titolo, bensì con la speranza che il mio lavoro possa contribuire, anche se in maniera infinitesimale, al progresso tecnologico.
\newpage

Lo scopo di questa tesi è quindi lo sviluppo di un'applicazione mobile geolocalizzata cross-platform per contesti di disaster-managent; o in altre parole la realizzazione di uno strumento per la segnalare la posizione delle varie emergenze nell'area colpita.\\
La presente tesi è così strutturata:
\begin{description}
\item [Capitolo 1:] In questo capitolo viene esposto il concetto di vulnerabilità e il contesto d'uso del sistema. 
\item [Capitolo 2:] In questo capitolo vengono illustrate le questioni etiche e i motivi tecnici affrontati nella scelta del map-provider.
\item [Capitolo 3:] In questo capitolo vengono riportati i requisiti di sistemade e alcune schermate significative dell'interfaccia grafica.
\item [Capitolo 4:] In questo capitolo si riportano i framework e alcuni dettagli implementativi dell'applicazione.
\end{description}