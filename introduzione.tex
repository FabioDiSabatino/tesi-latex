\chapter*{Introduzione}
\markboth{Introduzione}{Introduzione}
\addcontentsline{toc}{chapter}{Introduzione}
\thispagestyle{empty}

Durante una lezione del corso di progettazione di sistemi interattivi, tenuto dalla prof.ssa Laura Tarantino, fu chiesto a tutti gli studenti cosa ritenessero più tecnologico tra un libro e il proprio laptop. Una domanda apparentemente banale alla quale tutti gli studenti, me compreso, hanno risposto: laptop.\\
Ci venne spiegato che incosciamente siamo portati a ritenere un oggetto tecnologico se questo è stato inventanto dopo la nostra nascita o comunque in epoca recente.\\
Il fatto di non essere riuscito a rispondere correttamente a questa domanda mi ha fatto riflettere, come può un ingegnere non sapere per certo cosa sia tecnologico? Quando guardando la pubblicità dell'ultimo \textit{smartwatch} la risposta mi è parsa ovvia: negli ultimi decenni, il concetto di tecnologia è stato manipolato, trasformato e usato a fini puramente commerciali, inculcando nelle nostre menti un significato sbagliato di questo nobile concetto. Non ho nulla contro questi accessori e condivido la necessità di realizzare profitti; piuttosto contesto l'allocazione totale delle risorse e della forza lavoro nella sola ricerca/produzione di \textbf{tecnologia superflua}.
\textbf{La tecnologia è progresso, è l'insieme di tutti gli oggetti e studi che in qualche modo migliorano significativamente la nostra vita}; come ingegnere (aspirante), credo di dover contribuire a questo progresso (o almeno provarci).\\
Concludo dicendo che, non ho accettato questa tesi al fine unico di conseguire il titolo bensì con la speranza che il mio lavoro possa essere una \textbf{tecnologia} per il disaster-management e che più in generale contribuisca, anche se in maniera infinitesimale, al progresso tecnologico.
\newpage
Lo scopo di questa tesi è lo sviluppo di un'applicazione mobile cross-platform geolocalizzata, per contesti di disaster management.\\
Gli utenti possono segnalare la posizione delle emergerze e il proprio stato fisico; queste informazioni vengono inviate ad un main-server, il quale sarà utilizzato dai soccorritori per gestire il disastro e coordinare al meglio le operazioni di salvataggio. Inoltre attraverso l'applicazione è possibile visualizzare lo stato dei propri famigliari, dei luoghi d'interesse e quello generale dell'area colpita.\\
 Per realizzare l'applicazione si sono utilizzati i seguenti framework: Phonegap per renderla cross-platform, Ratchet per realizzare l'interfaccia e la libreria Javascript Leaflet per la mappa interattiva. Le posizioni degli utenti e degli eventi sono mappate in una griglia trasparente all'utente, utilizzando un algoritmo appositamente ideato.
La presente tesi è così strutturata:
\begin{description}
\item [Capitolo 1:] In questo capitolo viene esposto il concetto di vulnerabilità e il contesto d'uso del sistema. 
\item [Capitolo 2:] In questo capitolo vengono illustrate le questioni etiche e i motivi tecnici affrontati nella scelta del map-provider.
\item [Capitolo 3:] In questo capitolo vengono riportati i requisiti di sistema e alcune schermate significative dell'interfaccia grafica.
\item [Capitolo 4:] In questo capitolo si riportano i framework utilizzati e alcuni dettagli implementativi dell'applicazione.
\end{description}